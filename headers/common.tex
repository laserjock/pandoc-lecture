
%%%%%%%%%%%%%%%%%%%%%%%%%%%%%%%%%%%%%%%%%%%%%%%%%%%%%%%%%%%%%%%%%%%%%%%%%%%%%%%
% Written by Jordan Mantha
% Copyright 2013, 2016 Jordan Mantha <jordan.mantha@gmail.com>
%
%%%%%%%%%%%%%%%%%%%%%%%%%%%%%%%%%%%%%%%%%%%%%%%%%%%%%%%%%%%%%%%%%%%%%%%%%%%%%%%
%%%% Common Header
\usepackage{hyperref}
\setbeamertemplate{footline}{}
\useoutertheme{metropolis}
\useinnertheme{metropolis}
\usefonttheme{metropolis}
% Comment out below if using Owl color theme (white on black)
\usecolortheme{metropolis}
%%%%%%%%%%%%%%%%%%%%%%%%%%%%%%%%%%%%%%%%%%%%%%%%%%%%%%%%%%
%
%  Reset some default colors
%
\definecolor{darkred}{rgb}{0.7,0,0}
\definecolor{darkblue}{rgb}{0,0,0.8}
%\setbeamercolor{item}{fg=green}  %  The dot color
%\setbeamercolor{alert}{fg=red}
%\setbeamercolor{structure}{fg=blue}
%\setbeamercolor{frametitle}{fg=red}
%\setbeamercolor{description item}{fg=darkred!80!black}  %  Color of key word in desciption 
%
% owl color theme colors: red, green, blue, yellow, violet, brown, orange, cyan
%
%%%%%%%%%%%%%%%%%%%%%%%%%%%%%%%%%%%%%%%%%%%%%%%%%%%%%%
%
%  ALERT, STUCT and HUSH commands
%
% Adapted from BeamerLecture V1.1 by Don Eisenstein
% http://faculty.chicagobooth.edu/donald.eisenstein/more/BeamerLecture.html
%
% Usage:   Make \alrt{THIS RED}
%          The secret is \hush{cubs win}
%          The secret is revealed \hushalrt{in red}
%
\newcommand{\alrt}[1]{{\color{red} #1}}
\newcommand{\alrtl}[1]{{\color{red}\large #1}}
\newcommand{\alrtL}[1]{{\color{red}\Large #1}}
\newcommand{\struc}[1]{{\color{blue} #1}}
\newcommand{\strucL}[1]{{\color{blue}\Large #1}}
\newcommand{\strucl}[1]{{\color{blue}\large #1}}
%
\newcommand{\dred}[1]{{\color{darkred} #1}}
\newcommand{\dredl}[1]{{\color{darkred}\large #1}}
\newcommand{\dredL}[1]{{\color{darkred}\Large #1}}
%%%%%%%%%%%%%%%%%%%%%%%%%%%%%%%%%%%%%%%%%%%%%%%%%%%%%%
\newcommand{\hush}{\hushit}
%
\newcommand{\hushalrt}[1]{\hushit{{\color{red} #1}}}
\newcommand{\hushalrtl}[1]{\hushit{{\large\color{red} #1}}}
\newcommand{\hushalrtL}[1]{\hushit{{\Large\color{red} #1}}}
\newcommand{\hushstruc}[1]{\hushit{{\color{blue} #1}}}
\newcommand{\hushstrucl}[1]{\hushit{{\large\color{blue} #1}}}
\newcommand{\hushstrucL}[1]{\hushit{{\Large\color{blue} #1}}}
%
%%%%%%%%%%%%%%%%%%%%%%%%%%%%%%%%%%%%%%%%%%%%%%%%%%%%%%%%%%
%
%  Used to reset basic black for itemize/enumerates within certain environments
%
\newcommand{\blackcolors}{\setbeamercolor{itemize/enumerate body}{fg=black}%
\setbeamercolor{itemize/enumerate subbody}{fg=black}%
\setbeamercolor{itemize/enumerate subsubbody}{fg=black}}
%%%%%%%%%%%%%%%%%%%%%%%%%%%%%%%%%%%%%%%%%%%%%%%%%%%%%%%%%%
%
%  used with transstruc and transalert environment to shade hidden text
%
\newcommand{\cover}{\setbeamercovered{invisible}}
\newcommand{\semicover}{\setbeamercovered{transparent=30}}
%%%%%%%%%%%%%%%%%%%%%%%%%%%%%%%%%%%%%%%%%%%%%%%%%%%%%%%%%%
%
%  Try these instead of the itemize environments, that is:  
%
% \begin{ialert}
%   \item hello
%   \item hello again
% \end{ialert}
%
%  iitemize is a standard, to reveal successive items on successive slides
%
\newenvironment{iitemize}{\pause\begin{itemize}[<+->]}{\end{itemize}}
\newenvironment{ienumerate}{\pause\begin{enumerate}[<+->]}{\end{enumerate}}
\newenvironment{idescription}{\pause\begin{description}[<+->][.]}{\end{description}}
%
%  These others are fancier, higlightling the current item, making hidden
%   items translucent, etc..  Try them out to see. 
%
\newenvironment{ialert}{\pause\begin{itemize}[<+-| alert@+>]\blackcolors}{\end{itemize}}
\newenvironment{ishowalert}{\begin{itemize}[<1-| alert@+>]\blackcolors}{\end{itemize}} 
\newenvironment{itrans}{\semicover\pause\begin{itemize}[<+->]}{\end{itemize}\cover}
\newenvironment{itransalert}{\semicover\begin{itemize}[<+-| alert@+>]\blackcolors}{\end{itemize}\cover}
%%%%
